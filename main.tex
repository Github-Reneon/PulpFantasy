\documentclass[12pt, a4paper]{book}
\usepackage[width=4.375in, height=7.0in, top=1.0in, papersize={5.5in,8.5in}]{geometry}
\usepackage{indentfirst}
\usepackage[utf8]{inputenc}
\usepackage[T1]{fontenc}
\usepackage{ebgaramond}
\usepackage{imakeidx}

\usepackage{fancyhdr}
\usepackage{titlesec}
\usepackage{poemscol}

\pagestyle{fancy}
\fancyhf{}
\rhead{\thepage}
\lhead{\leftmark}
\title{Brightblade}
\author{Andrew Lovick}
\date{}

\makeindex


\begin{document}
\pagenumbering{gobble}
\maketitle
\clearpage
\tableofcontents
\printindex
\clearpage

\pagenumbering{arabic}

\chapter{Adventurers}
The world of Mundthir is a boon for some, a curse for others; but I'd say it's best for none other than the adventurer. 

Yes, the adventurer! That plucky man or woman, who travels Mundthir and its vast continents, and kingdoms that sit upon it, in order to seek fortune, fame and all of that gubbins.

I don't mean to fully put down the adventurer. There have been many occasions where the noblest amongst them have used their skills and sometimes even their life to save Mundthir from dark threats. What's more if I was to put down adventurers as a whole, I'd be putting down myself!

Aye reader, that's right. I was an adventurer starting near one-hundred and thirty two years ago, only stopping in the last decade or two. It's been a long and interesting life all told, with many an adventure to call my own. That being said (and most half elves my age wouldn't admit it) our fey blood doesn't stop this feeling of thinness in your soul, like you've cheated the reaper decade after decade. I'm sure after all I've been through he'll have some choice words for me.

I apologise dear reader. Adventurers. They used to be a copper piece in a horde. Common as muck. And there was good reason for it; all across Mundthir, there's ageless bounty to be found for those with a death wish to look for it. 

Case in point, I once knew an adventurer named Trimst. She was a lovely barbarian lady from the Gorthian Steppes, some north east way of Noth. Being in so close proximity to that intolerable and black kingdom she rode east near Priaethier, a small republic that unfortunately doesn't exist any more, but where I was adventuring myself at the time, in order to find a weapon that would be able to defend her people.

She knew of the Brightblade. A weapon of shimmering gold, and a magical heat that was of legend. The same legends said it was the property of the old emperor of Tet-Tehtu, a almost obscure figure from an obscure empire by the time of writing; but such legends passed by oral and musical tradition in Gorthia kept the knowledge alive in ways our written southern tradition could not. It's funny how that tends to be the case.

We'd met in a Priaethierin tavern in the capital, the Lost Dog. I'd travelled to Ars Priaether after taking my leave from some small incident which \textit{I will not} go into, and she'd been the first thing that had caught my eye. Not to toot her horn too much, that is to say; the denizens of the Lost Dog looked hapless and forlorn to be sure; but she still stood out like a flame in the dark.

Trimst was muscular, pretty, and bubbly; with bouncing blonde hair that swept around as quickly as her eyes did. And that's what drew me to her ultimately, her eyes; for if she didn't have those wary, suspicious eyes so characteristic of someone who's fought many battles, then I'd have just written her off as some runaway.

So we sat and chatted, and much a credit to her Gorthian heritage she boasted of past fights, and past drinks. As we had tankard of ale after ale the stories turned to legends, and legends to songs, and songs to simply waving her sword around challenging any fool to a fight. Thankfully there were no fools to be found, and so no fight to be had, a fact that genuinely saddened her.

When the mood died, she told me of her quest, and how she'd come to know of it. That she was to look for the Tet-Tehtan tomb of the old emperor. I knew a little of that old culture though, and their habit of using traps. I warned her that I was seeking gold in a Tet-Tehtan crypt a decade ago, and that the dangers posed were not that of any simple dungeon to be raided.

But she laughed her iconic throaty laugh, almost as if putting on for a show, and told me;

``Aye, but caution is for those who plan to grow old my friend. I plan to live forever through legend and song!''

As I wrote before dear reader... a death wish. 

But so enraptured by her spirit and finding common cause with our hatred of the kingdom of Noth, we decided to band together. I rationalised it by thinking that I was stronger now, smarter, and a lot more careful. There was bound to be hoards of treasure in a Tet-Tehtan tomb, all of which would likely go to myself since Trimst only wanted the sword. 

We made plans as all good adventurers did. We agreed on a split to the loot (as I suspected all she wanted was the fame and the sword, which was fine by me), but then it dawned on me. Where were we to go? At the question Trimst took a deep drink on her ale, grinning like the drunk barbarian she was.

``I know exactly where we're going chum!'' She exclaimed. ``We're headed where the songs lead us!''

She got onto her stool, and wobbling she stood upright. Once she'd found her balance, she began singing;

%https://tex.stackexchange.com/questions/160560/package-to-typeset-poems

\begin{poem}
\begin{stanza}
``We wish you best in every age\verseline
We sing of clashes, slashes, steel\verseline
Warriors filled with mirth and zeal\verseline
Of tales of old, and fates ere sealed\verseline
And only one who makes us feel;\verseline
That's our Ulduar Brokensage!
\end{stanza}
\begin{stanza}
He travelled far and travelled south\verseline
Priaethierin woods and further too\verseline
Where mountains travel two by two\verseline
And the river Ai-Crans flows through\verseline
To find the Valley of Tehtu\verseline
And to the river's mouth
\end{stanza}
\begin{stanza}
And there he found a castle dour\verseline
Of marble jewels and useless things\verseline
He ripped the draw bridge off its hinge\verseline
He climbed the rampart swift as wings\verseline
Drank and then began to sing\verseline
And drank more by the hour
\end{stanza}
\begin{stanza}
Ulduar raided the castle larder\verseline
A warrior with eyes so cold\verseline
And wearing armour made of gold\verseline
`Screamed leave this place you upright gnoll!'\verseline
And struck our hero, a move too bold\verseline
So Ulduar struck back harder!
\end{stanza}
\begin{stanza}
Then with every clash of sharpness made\verseline
Ulduar's blade began to crack\verseline
`Sunder' his sword could not attack\verseline
Through the defence and the tact\verseline
`How can you beat me Tet-Tehtac?'\verseline
`You face the might of Brightblade!'
\end{stanza}
\begin{stanza}
Ulduar fell back with a thud\verseline
Sunder cracked in twain\verseline
All he felt was pain\verseline
Nere was he seen again\verseline
But memories we retain\verseline
In our songs and Garthian blood!
\end{stanza}
\end{poem}

It was foolish, I know. It seemed I had a death wish too. I'm sure Mr Reaper was tapping his foot impatiently as he saw the two of us leave the tavern to leave the city that night, to find Brightblade.

\chapter{Interdiction}

We slept that night in the woods, having made blindly and drunkenly out of the city. I thank the Three every morn for the protection they gave to us; stumbling in the dark like idiots. I woke up under an old pine, covered in its leaves and the morning dew. Wiping my face and sitting up, Trimst lay in the middle of the clearing we stopped at, illuminated by the morning sun as if by limelight at the plays.

Spread like a starfish and flat on her back, she snored loudly. That was something I'd come to know in the following days. And not just her snoring, but she did everything loudly; boasted the loudest, sang the loudest, couldn't even sneak a damn especially in the chain-mail she loved to wear at all times.

Seeing as I was first up, I decided to take stock. My rations, 

\end{document}